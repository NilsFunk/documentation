\section{Fundamental Dynamic Programming}

\subsection{Graph Search Algorithms}

\subsubsection{RRT}

\subsubsection{RRT*}

\subsubsection{A*}

\textbf{Reference:} \\
\href{https://www.redblobgames.com/pathfinding/a-star/}{Red Blob Games - A*}

\href{http://theory.stanford.edu/~amitp/GameProgramming/AStarComparison.html}{Red Blob Games - A* Comparison}


\subsubsection{Dijkstra}

\subsubsection{Jump Point Search}

In computer science, Jump Point Search (JPS) is an optimization to the A* search algorithm for \textbf{uniform-cost grids}. It reduces symmetries in the search procedure by means of graph pruning, eliminating certain nodes in the grid based on assumptions that can be made about the current node's neighbors, as long as certain conditions relating to the grid are satisfied. As a result, the algorithm can consider long "jumps" along straight (horizontal, vertical and diagonal) lines in the grid, rather than the small steps from one grid position to the next that ordinary A* considers.

Jump point search preserves A*'s optimality, while potentially reducing its running time by an order of magnitude.

\textbf{Reference:} \\
\href{https://zerowidth.com/2013/05/05/jump-point-search-explained.html}{Zero Width - Jump point search explained}

\href{https://harablog.wordpress.com/2011/09/07/jump-point-search/}{Hara Blog - Jump point search}

\href{https://www.gamedev.net/articles/programming/artificial-intelligence/jump-point-search-fast-a-pathfinding-for-uniform-cost-grids-r4220/}{Game Dev - Jump point search fast a pathfinding for uniform-cost grids}

\href{https://gamedevelopment.tutsplus.com/tutorials/how-to-speed-up-a-pathfinding-with-the-jump-point-search-algorithm--gamedev-5818}{Game Development - How to speed up a pathfinding with the jump point search algorithm}






