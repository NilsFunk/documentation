\section{Trajectory Planning}

\subsection{Liu - Planning Dynamically Feasible Trajectories for Quadrotors Using Safe Flight Corridiors in 3-D Complex Environments}

\subsubsection{Contribution}
A method to formulate trajectory generation as a quadratic program (QP) using the concept of a Safe Flight Corridor (SFC)

\subsubsection{Summary}
The SFC is a collection of convex overlapping polyhedra that models free space and provides a connected path from the robot to the goal position. 

Derive an efficient convex decomposition method that builds the SFC from a piece-wise linear skeleton obtained using a fast graph search technique.

The SFC provides a set of linear inequality constraints in the QP allowing real-time motion planning. 

Develop a framework of Receding Horizon Planning, which plans trajectories within a finite footprint in the localmap, continuously updating the trajectory through a re-planning process. 

\textbf{Pipeline:}

A. Path Planning\\

Map representation: Occupancy grid\\
Graph search algorithm: Jump Point Search (JPS)\\

B. Safe Flight Corridor Construction\\

Path $P = \left\langle \mathbf{p}_0, \mathbf{p}_1, ... ,\mathbf{p}_n,  \right\rangle$\\
Line segement $L_i = \left\langle \mathbf{p}_i \rightarrow \mathbf{p}_{i+1} \right\rangle$\\
Flight corridor $C_i$\\

Generate a convex polyhedron around each line segement $L_i$ in Path $\mathbf{P}$ to construct a valid $SFC(\mathbf{P})$.\\

One criterion for the construciton of the SFC is that two consecutive polyhedra need to intersect in a non-empty set containing $\mathbf{p}_{i+}$.\\

- Use bounding box to confine the space around $L_i$ and to reduce computation time\\
- Use shrinking process to guarantee that a non-point robot is collison-free\\

Steps:\\
1. Find ellipsoid that does not include any obstacle points $O$\\
Ellipsoid $\xi$ axis $a,b,c$\\
- Ellipsoid is a sphere centered at the mid point of $L_i$ and with a diameter equal to the length of $L_i$\\
- Assume the length of ellipsoid's $\tilde{x}$-axis is fixed and aligned with $L$\\
- Reduce lenth of the other two axes until the spheroid contains no obstacles\\

1.1 Shrink an initial sphere to derive the maximal spheroid (an ellipsoid with two axes of qual length\\
Search for closest obstacle $\mathbf{p}*$ in $O$ form the center of the the ellipsoid $\xi$\\
The line segment $L_i$, defines the plane of $\tilde{x}-\tilde{y}$ axes of the spheroid\\

1.2 Stretch this spheroid along the third axis to obtain the final ellipsoid\\
Strech the length of the $\tilde{z}$-axis of the spheroid to make it equal to $a$ to form a new initial ellipsoid.\\

The actual value of c can be determined through finding another closest point using the similar process.\\
 
2. Find polyhedron\\
Construct the polzhedron $C_i$ from tangent plances to a sequence of dilated ellipsoids.\\

Iteratively dilate the ellipsoid until it is in contact with the next obstacle in point $\mathbf{p}_i$ and creat a half space $H_i = \lbrace \mathbf{p} \rvert \mathbf{a}_i^T\mathbf{p} < b_i \rbrace$ and remove the obstacles that lie outside of $H_i$.\\

3. Bounding box\\
The bounding box for $L$ is composed of 6 rectangles such that the axis of the bouding box is aligned with $L$ and the minimum distance from each face to $L$ is $r_s$.\\

$r_s = \geq \frac{v_{max}^2}{2a_{max}^2}$\\

4. Shrink\\
Use the original map $M$ to generate the SFC and shrink the SFC by the robot radius $r_r$ in order to guarantee safety. The shrinking process is applied by pushing every support hyperplane by $r_r$.\\

If the minimum distance $d(L_i,H_{i,j}') < r_r$ (i.e. the line segment $L_i$ is outside the shrunk polyhedron $H'$) the normal of the hyperplane gets adjustet s.t. $d(L_i,H_{i,j}') = r_r$.\\

C. Trajectory Optimiazation\\

Convex optimization for minimum snap trajectories is formed as a quadratic program with constraints for each trajectory segement $\Phi$\\

$\frac{d^k}{dt^k}\Phi_i(\Delta t_i) = \frac{d^k}{dt^k}\Phi_{i+1}(0), k=0...4$ \\

$\mathbf{A}_i^T \Phi_i (t) < \mathbf{b}$ \\

here the matrices $\mathbf{A}_i, \mathbf{b}_i$ correspond to the $i$-th polyhedron $C_i$\\

Time allocation:\\
Similar to [Richter] \\

$\Delta t_i'=max \lbrace 1, (\frac{v_{max}}{\overline{v}_{max}}), (\frac{a_{max}}{\overline{a}_{max}})^\frac{1}{2}, (\frac{j_{max}}{\overline{j}_{max}})^\frac{1}{3} \rbrace \Delta t_i$\\

D. Receing Horiyon Planning (RHP)\\

Solve optimal control problem over a fixed future time interval with a planning horizon of distance $d_r$. The generation of the trajectory for the next epoch is started when the robot is executing the trajectorz at the current epoch. Choose execution time $T_e$ such that the time to generte a new trajectory is guaranteed to be less. Since the execution time $T_e$ is bigger than the time it takes for generating a trajectory, the robot is always able to transit to track a new trajectory when it  finishes executing the current one. 

\subsubsection{Results}
The re-planning process takes between 50 to 300 ms for a large and cluttered map. Approach is feasible, complete, with applications to high-speed flight in both simulated and physical experiments using quadrotors.

\subsubsection{Flaws}
In our optimization process, we use a sample-based method
to confine each polynomial, the details for which can be found in [2]
\\

\textbf{Note}: The optimization process seems still to just sample the trajectory at certain points, rather than to check if the trajectory segement is within the flight corridor at all times.

\subsubsection{Highly Related Work}
\begin{itemize}
\item Trajectory planning: Derivation of UAV differential flatness and optimization based trajectory gernation of minimum snap trajectories with corridor constraints
\\
Mellinger - Minimum snap trajectory gerneation and control of quadrotors

\item Trajectory planning: 
\\
Richter - Polynomial trajyectory planning for aggressive quadrotor flight in dense indoor environements

\item Trajectory planning:
\\
Chen - Online generation of collison-free trajectories for quadrotor flgit in unknown cluttered environments

\item Trajectory planning:
\\
Liu - High speed navigation for quadrotors with limited onboard sensing

\item Trajectory planning:
\\
Watterson - Safe receding horiyon control for aggressive MAV flight with limited range sensing

\item Trajectory planning:
\\
Mellinger - Trajectory genration and control for quadrotors

\item Graph Search Algorithm: Jump Point Search - Fast search strategy in uniform-cost grids that can speed up A* by an order of magnitude
\\
Harabor - Online graph pruning for path finding on grid maps

\end{itemize}

\subsubsection{Covered Topics}
Quadratic program, convex polyhedra, convex decompositon method, linear inequality constraints, receding horizon planning (RHP), trajectory parameterization, mixed integer methods, convex free space, generic, parallelopipeds, A*, Dijkstra, jump point search (JPS), natural/purned neighbors, symmetric positive definit, ellipsoid, half space, non-linear control


\subsection{Chen - Online Generation of Collision-Free Trajectories for Quadrotor Flight in Unknown Cluttered Environments}

\subsection{Mellinger - Minimum snap trajectory gerneation and control of quadrotors}

\subsection{Richter - Polynomial trajyectory planning for aggressive quadrotor flight in dense indoor environements}

\subsection{Chen - Online generation of collison-free trajectories for quadrotor flgit in unknown cluttered environments}

\subsection{Liu - High speed navigation for quadrotors with limited onboard sensing}

\subsection{Watterson - Safe receding horiyon control for aggressive MAV flight with limited range sensing}

\subsection{Mellinger - Trajectory genration and control for quadrotors}

\subsection{Harabor - Online graph pruning for path finding on grid maps}

\subsubsection{Contribution}
Jump Point Search - Fast search strategy in uniform-cost grids that can speed up A* by an order of magnitude
